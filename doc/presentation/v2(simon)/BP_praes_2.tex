\documentclass[accentcolor=tud9d,colorbacktitle,inverttitle,landscape,german,presentation,t]{tudbeamer}
\usepackage{ngerman}
\usepackage[T1]{fontenc}
\usepackage[latin1]{inputenc}
\usepackage{helvet}
\usepackage{graphicx}
\usepackage{picins}


\begin{document}

	% Titel
	\title{Instant Message Whispering via Covert Channels}
	\subtitle{Simon Kadel, Gruppe 35 }
	
 	% Fu�zeile
	\author{S. Kadel, J.S. Bunten, M.S. Oehler, A.S. St�hlmeier}
	\date{3. Dezember 2013}
	\logo{\includegraphics{../Bilder/tklogo.jpg}}

\begin{titleframe}
	\begin{figure}
	\centering
	\includegraphics[scale=.55]{../Bilder/InstantMessenger.jpg}
	\end{figure}
\end{titleframe}

%\begin{frame}
%	\frametitle{Einf�hrung}
%	
%	\begin{minipage}{.55\linewidth}
%	Verstecktes Instant Messaging
%	\begin{itemize}
%		\item Senden von Nachrichten durch Firewalls
%		\item Vermeiden von Entdeckung, Abh�ren und Abfangen
%	\end{itemize}	
%	
%	\end{minipage}
%	\hfill
%	\begin{minipage}{.4\linewidth}
%	{\includegraphics[scale=.3]{../Bilder/Firewall.jpg}}
%	\end{minipage}
%	
%	\begin{description}
%		\item {}
%		\item{}
%		\item <2>\textbf{Kryptographie} Verschl�sselt die Daten, Kanal bleibt sichtbar
%		\item<2>\textbf{Covert Channels} Versteckt den Kanal
%	\end{description}
%	
%\end{frame}

\begin{frame}

	\frametitle{Projektbeschreibung}
	\textbf{Cover Channel } Kommunikation kann nicht bemerkt werden\\
	\vspace{.5cm}	
	\only<2>{
	Programmieren einer Bibliothek f�r Covert Channels
	\begin{itemize}
		\item �ffnet und verwendet Covert Channels
		\item Implementierung eines Frameworks
		\item konkrete Covert Channels als Plugin
		\item Ver�ffentlichung als Open Source
	\end{itemize}}
	\begin{center}
	%\parpic[r]{\includegraphics[scale=0.25]{Bilder/framework-puzzle.png}}
	\end{center}

\end{frame}

\begin{frame}
	\frametitle{Top-Level Design}
		\begin{figure}
			\centering
			\includegraphics[scale=.4]{../Bilder/top-level-design.png}
		\end{figure}
\end{frame}

\begin{frame}

	\frametitle{QS-Ziele und ihre Sichherstellung}
	\begin{itemize}
%		\item <1->\textbf{Modularit�t} \\
%		\only<2>
%		{\begin{itemize}
%			\item Modulares Design des Framework
%			\item Feste, dokumentierte Modulschnittstellen
%		\end{itemize}}
		\item <1->\textbf{Zuverl�ssigkeit} \\
		\only<2>
		{\begin{itemize}
			\item automatisierte Tests mit Boost.Test
			\item Ticket-System im SCM-Server
			\item Code Reviews
		\end{itemize}}
		\item <1->\textbf{Testbarkeit}	\\
		\only<4>
		{\begin{itemize}
			\item Zuteilung der Aufgaben an Module
			\item eigene Tests schreiben
			\item Architektur bei Problemen anpassen
		\end{itemize}}
	\end{itemize}

\end{frame}

\begin{frame}
	\frametitle{Zeitaufwand}
	\begin{center}
		\includegraphics[scale=0.43]{../Bilder/zeitsummen2.png} \newline
	\end{center}

\end{frame}

\begin{frame}
\frametitle{Zeitmanagement}
		\begin{figure}
			\centering
			\includegraphics[scale=.4]{../Bilder/iteration2.png}
		\end{figure}
		\begin{figure}
			\centering
			\includegraphics[scale=.4]{../Bilder/velocity2.png}
		\end{figure}
\end{frame}

\begin{frame}
\frametitle{Zusammenfassung}
	\vspace{.7cm}
	\hfill
	\begin{minipage}{.85\linewidth}
	\begin{itemize}
		\item Covert Channels zur versteckten Kommunikation
		\item Entwicklung einer modularen Bibliothek 
		\item nutzbar f�r m�glichst viele Covert Channels
	\end{itemize}
	\end{minipage}
\end{frame}


\end{document}